\chapter{Software-Architektur}

\section{Frontend}
Durch eCourse soll es \gls{Studierende}n und \gls{Dozierende}n möglich sein Dateien untereinander auszutauschen. Ebenso sollen Verwaltungsangestellte einen Einblick in die ausgetauschten Dateien haben und verwaltende Tätigkeiten ausführen können. Diese drei Benutzergruppen greifen direkt auf die Anwendung eCourse zu. Es gibt keine weiteren Schnittstellen zu anderen Anwendungen, wie in Kapitel \ref{sec:kontext} bereits erläutert wurde. 

\section{ERM-Diagramm}
Blickt man genauer in die Anwendung eCourse hinein, ergibt sich eine Aufteilung in \gls{Backend} und \gls{Frontend}.

\subsection{UML-Diagramm}
Das \gls{Frontend} beschreibt dabei den Teil der Anwendung, der näher am Nutzer liegt, es interagiert mit dem Benutzer. Es ist also die Schnittstelle zwischen Nutzer und der dahinter liegenden Logik der Anwendung. Alle drei Nutzergruppen haben Zugriff auf das \gls{Frontend}, wobei das \gls{Frontend} abhängig von der Nutzergruppe unterschiedlich mit dem Benutzer interagiert. Eine genaue Darstellung der Rechte die die jeweilige Nutzergruppe hat ist in \ref{fib:erlaubnis} zu finden.
